%-----------------------------------------------------------------------------%
\chapter{\babEnam}
%-----------------------------------------------------------------------------%
Bab ini menerangkan kesimpulan akhir dari penelitian berdasarkan hasil analisis, pembahasan dan penyusunan skripsi yang telah dilakukan. Selain itu juga dicantumkan juga saran untuk penelitian selanjutnya.
%---------------------------------------------------------------
\section{Kesimpulan}
Dari penelitian ini dapat ditarik kesimpulan sebagai berikut.
\begin{enumerate}
	\item Berdasarkan hasil penelitian dengan menerapkan instrumen penelitian, diketahui bahwa salah satu \f{website} \f{e-government} yaitu \f{website} Direktorat Jenderal Pajak Indonesia yang menjadi portal informasi perpajakan bagi masyarakat Indonesia serta akses penggunanya yang cukup tinggi masih belum memenuhi standar kriteria \f{website e-government} yang baik, sesuai dengan komponen \f{usability heuristic} dan \f{g-quality}. 
	\item Diketahui juga pada penelitian bahwa \f{website e-government} Direktorat Jenderal Pajak India lebih baik dari \f{website} Direktorat Jenderal Pajak Indonesia pada semua komponen evaluasi \f{usability heuristic} dan \f{g-quality} untuk \f{website e-government}.
	\item Negara yang memiliki \f{ranking website e-government} lebih unggul dari negara lain belum tentu memiliki nilai evaluasi \f{usability} yang lebih baik.
	\item Didapatkan data beserta analisis perbandingan yang menghasilkan rekomendasi perbaikan untuk \f{website} Direktorat Jenderal Pajak Indonesia dengan menggunakan komponen \f{usability heuristic} dan \f{g-quality}.
	\item Dikembangkan sebuah \f{clickable-prototype} dari hasil rekomendasi dan analisa data yang menjadi bahan masukan untuk penyelenggara \f{e-government}, khususnya Direktorat Jenderal Pajak Indonesia.
\end{enumerate}
%---------------------------------------------------------------
\pagebreak
%---------------------------------------------------------------
\section{Saran}
Adapun saran untuk penelitian selanjutnya adalah sebagai berikut:
\begin{enumerate}
	\item Penelitian ini menggunakan studi kasus \f{website} Direktorat Jenderal Pajak antara Indonesia dengan India. Evalulasi \f{usability e-government} dapat menggunakan \f{website e-government} lain misalnya pada lingkup yang regional agar mendapatkan hasil evaluasi yang lebih luas. 
	\item Penelitian ini dilakukan menggunakan \f{usability testing} dengan metode survei. Penelitian selanjutnya dapat digunakan metode lain dalam pengumpulan data, seperti metode \f{heuristic evaluation}, \f{focus group}, \f{AB Testing} dan lain-lain karena setiap metode memberikan hasil pengujian yang berbeda dan memiliki kualitas pengujian masing-masing.
	\item Penelitian ini mencakup \f{website e-government} yang dimiliki oleh kementrian pemerintah Indonesia. Penelitian selanjutnya diharapkan menguji lebih banyak lagi \f{website} dari pelayanan \f{e-government} untuk meningkatkan kualitas \f{website-website e-government} di Indonesia.
	\item Usahakan meminimalkan gangguan teknis seperti koneksi \f{internet} atau \f{server} yang terputus saat pengujian dengan responden karena memengaruhi komponen penilaian pada \f{usability testing}.
	\item \f{High-fidelity prototype} yang dikembangkan seharusnya dievaluasi oleh responden untuk mendapatkan hasil yang lebih baik.
\end{enumerate}
%---------------------------------------------------------------
