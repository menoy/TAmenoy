%-----------------------------------------------------------------------------%
\chapter{\babEnam}
%-----------------------------------------------------------------------------%
Bab ini menerangkan kesimpulan akhir dari penelitian berdasarkan hasil analisis, pembahasan dan penyusunan skripsi yang telah dilakukan. Selain itu juga dicantumkan juga saran untuk penelitian selanjutnya.
%---------------------------------------------------------------
\section{Kesimpulan}
Dari penelitian ini dapat ditarik kesimpulan sebagai berikut.
\begin{enumerate}
	\item Telah dirancang instrumen penelitian untuk mengevaluasi aspek \f{usability} dan \f{g-quality} dari \f{website e-government}.
	\item Berdasarkan hasil penelitian, diketahui bahwa salah satu \f{website} pemerintahan yaitu \f{website} Direktorat Jenderal Pajak Indonesia yang menjadi portal informasi perpajakan bagi masyarakat Indonesia masih belum memenuhi standar kriteria \f{website e-government} yang baik, sesuai dengan komponen \f{usability heuristic} dan \f{g-quality}. 
	\item \f{Website e-government} Direktorat Jenderal Pajak India lebih unggul dari \f{website} Direktorat Jenderal Pajak Indonesia pada semua komponen evaluasi \f{usability heuristic} dan \f{g-quality} untuk \f{website e-government}.
	\item Negara yang memiliki \f{ranking e-government} lebih tinggi dari negara lain belum tentu memiliki nilai evaluasi \f{usability} yang lebih baik.
	\item Didapat rekomendasi perbaikan \f{website} Direktorat Jenderal Pajak Indonesia dari hasil evaluasi perbandingan \f{website e-government} dengan menggunakan komponen \f{usability heuristic} dan \f{g-quality}.
	\item Dikembangkan sebuah \f{clickable-prototype} dari hasil rekomendasi dan analisa data yang menjadi bahan masukkan untuk penyelenggara \f{e-government}, khususnya Direktorat Jenderal Pajak Indonesia.
\end{enumerate}
%---------------------------------------------------------------
\pagebreak
%---------------------------------------------------------------
\section{Saran}
Adapun saran untuk penelitian selanjutnya adalah sebagai berikut:
\begin{enumerate}
	\item Penelitian sebaiknya menggunakan waktu yang cukup panjang. Hal ini disebabkan karena birokrasi untuk pengumpulan data dengan populasi dan sampel yang \f{purposive} membutuhkan mekanisme yang cukup rumit. 
	\item Penelitian dilakukan dengan populasi yang lebih luas lagi, tidak hanya target pengguna saja yang dijadikan responden, namun masyarakat yang lebih luas karena pelayanan \f{e-government} harus mencapai semua kalangan.
	\item Penelitian ini menggunakan studi kasus \f{website} Direktorat Jenderal Pajak antara Indonesia dengan India. Sesungguhnya evalulasi \f{usability e-government} dapat menggunakan \f{website e-government} lainnya pada lingkup yang lebih regional. 
	\item Penelitian ini dilakukan menggunakan \f{usability testing} dengan metode survei. Penelitian selanjutnya dapat digunakan metode lain dalam pengumpulan data, seperti metode \f{heuristic evaluation}, \f{focus group} dan lain-lain.
	\item Penelitian ini mencakup \f{website e-government} yang dimiliki oleh departemen pemerintahan. Penelitian selanjutnya diharapkan menguji lebih banyak lagi elemen dari pelayanan \f{e-government}.
	\item Usahakan meminimalkan gangguan teknis seperti koneksi \f{internet} atau \f{server} yang terputus saat pengujian dengan responden.
	\item \f{Prototype} perlu dievaluasi beberapa kali kepada responden untuk mendapatkan hasil yang memuaskan.
\end{enumerate}
%---------------------------------------------------------------
