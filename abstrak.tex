%
% Halaman Abstrak
%
% @author  Andreas Febrian
% @version 1.00
%

\chapter*{Abstrak}

\vspace*{0.2cm}

\noindent \begin{tabular}{l l p{10cm}}
	Nama&: & \penulis \\
	Program Studi&: & \program \\
	Judul&: & \judul \\
\end{tabular} \\ 

\vspace*{0.5cm}

\noindent 
Perkembangan teknologi internet di dunia mendukung lahirnya pemanfaatan layanan \f{e-government} kepada masyarakat. Satu dekade lebih lamanya pelayanan \f{e-government} di Indonesia telah tersedia. Namun, pelayanan tersebut belum dapat diketahui kualitasnya dari aspek \f{usability} dan \f{user experience}. Dilakukanlah evaluasi \f{usability website e-government} kepada Direktorat Jenderal Pajak Indonesia selaku salah satu pelaksana \f{e-government} di Indonesia. Evaluasi dilakukan dengan membandingkan aspek \f{usability heuristic}, kepuasan pengguna dan \f{g-quality} yang dihitung efisiensi dan efektivitasnya terhadap \f{website} DJP India. Dari hasil evaluasi diketahui bahwa \f{website} DJP Indonesia belum memenuhi nilai baik yang dibutuhkan sebuah \f{website e-government}. Dari evaluasi tersebut, dikembangkanlah \f{prototype} rekomendasi perbaikan untuk DJP Indonesia.
\vspace*{0.2cm}

\noindent Kata Kunci: \\ 
\noindent Direktorat Jenderal Pajak, \f{E-government}, \f{g-Quality}, Kepuasan Pengguna, \f{Usability}, \f{User Experience}, \f{Usability Testing}, \f{Web-based}\\ 

\newpage