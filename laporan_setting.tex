%-----------------------------------------------------------------------------%
% Informasi Mengenai Dokumen
%-----------------------------------------------------------------------------%
% 
% Judul laporan. 
\var{\judul}{Pengembangan \f{Prototype User Interface} untuk Membandingkan Aspek Usability dan Fungsionalitas dari Website \f{E-Government}: Studi Kasus Website Direktorat Jenderal Pajak Indonesia dengan India}
% 
% Tulis kembali judul laporan, kali ini akan diubah menjadi huruf kapital
\Var{\Judul}{Pengembangan \f{Prototype User Interface} untuk Membandingkan Aspek Usability dan Fungsionalitas dari Website \f{E-Government}: Studi Kasus Website Direktorat Jenderal Pajak Indonesia dengan India}
% 
% Tulis kembali judul laporan namun dengan bahasa Ingris
\var{\judulInggris}{User Interface Prototype Development to Compare Usability and Functional Aspect in E-Government Website: Case Study Indonesia with India Tax Department Website}

% 
% Tipe laporan, dapat berisi Skripsi, Tugas Akhir, Thesis, atau Disertasi
\var{\type}{Skripsi}
% 
% Tulis kembali tipe laporan, kali ini akan diubah menjadi huruf kapital
\Var{\Type}{Skripsi}
% 
% Tulis nama penulis 
\var{\penulis}{Aldi Reinaldi}
% 
% Tulis kembali nama penulis, kali ini akan diubah menjadi huruf kapital
\Var{\Penulis}{Aldi Reinaldi}
% 
% Tulis NPM penulis
\var{\npm}{1106022780}
% 
% Tuliskan Fakultas dimana penulis berada
\Var{\Fakultas}{Ilmu Komputer}
\var{\fakultas}{Ilmu Komputer}
% 
% Tuliskan Program Studi yang diambil penulis
\Var{\Program}{Ilmu Komputer}
\var{\program}{Ilmu Komputer}
\var{\programEng}{Computer Science}
% 
% Tuliskan tahun publikasi laporan
\Var{\bulanTahun}{Juni 2015}
% 
% Tuliskan gelar yang akan diperoleh dengan menyerahkan laporan ini
\var{\gelar}{Sarjana Ilmu Komputer}
% 
% Tuliskan tanggal pengesahan laporan, waktu dimana laporan diserahkan ke 
% penguji/sekretariat
\var{\tanggalPengesahan}{21 Juni 2013} 
% 
% Tuliskan tanggal keputusan sidang dikeluarkan dan penulis dinyatakan 
% lulus/tidak lulus
\var{\tanggalLulus}{5 Juli 2013}
% 
% Tuliskan pembimbing 
\var{\pembimbing}{Dadan Hardianto, S.Kom, M.Kom}
\var{\pembimbingdua}{}
% 
% Alias untuk memudahkan alur penulisan paa saat menulis laporan
\var{\saya}{Penulis}

%-----------------------------------------------------------------------------%
% Judul Setiap Bab
%-----------------------------------------------------------------------------%
% 
% Berikut ada judul-judul setiap bab. 
% Silahkan diubah sesuai dengan kebutuhan. 
% 
\Var{\kataPengantar}{Kata Pengantar}
\Var{\babSatu}{Pendahuluan}
\Var{\babDua}{Tinjauan Pustaka}
\Var{\babTiga}{Metodologi Penelitian}
\Var{\babEmpat}{Analisis Kebutuhan dan Pendefinisian \textit{Requirement}}
\Var{\babLima}{\textit{Prototyping} dan Evaluasi}
\Var{\babEnam}{Penutup}
\Var{\babTujuh}{}
