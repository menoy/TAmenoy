%-----------------------------------------------------------------------------%
\chapter*{\kataPengantar}
%-----------------------------------------------------------------------------%
\f{Alhamdulillahirabbil'alamin}, segala puji dan syukur kehadirat Allah SWT, karena atas hidayah dan rahmat-NYA, penulis dapat menyelesaikan skripsi ini. Tak lupa sholawat serta salam tak henti-hentinya penulis panjatkan kepada Rasulullah SAW, karena atas jasa beliau penulis dapat menikmati indahnya agama ISLAM yang sempurna ini. Penulisan skripsi ini ditujukan untuk memenuhi salah satu syarat dalam menyelesaikan program \gelar, Universitas Indonesia. Penulis sadar bahwa dalam proses penulisan skripsi ini penulis tidak sendirian. Sebab tanpa bimbingan dan dukungan berbagai pihak, penulis mungkin tidak akan berhasil menyelesaikan penulisan skripsi ini. Maka dari itu, penulis ingin menyampaikan ucapan terimakasih banyak kepada: 
\begin{enumerate}
	\item Kedua orang tua penulis dan juga keluarga penulis, Bapak Kasad Eka Saputra dan Ibu Eem Rochemah, kepada Nana Rosdiana, Atik Kartika, Tanti Santina, Deden Sultonik dan Yayah Hidayah atas doa restu dan dukungannya baik moril maupun materil yang tiada hentinya diberikan kepada penulis. Terimakasih atas segala doa dan dukungannya sehingga penulis dapat meraih dan menjalankan tingkat pendidikan tinggi di Universitas Indonesia ini.
	\item Bapak Dadan Herdianto, selaku dosen pembimbing yang telah banyak meluangkan waktu, tenaga dan pikiran untuk membimbing penulis sehingga skripsi ini dapat di selesaikan pada waktunya.
	\item Bapak Setiadi Yazid, selaku pembimbing akademik yang terus mendukung, membimbing dan membina penulis selama menempuh kuliah di Fakultas Ilmu Komputer Universitas Indonesia ini.
	\item Teman Adkesma BEM Fasilkom UI 2012 \& 2013 Korbid dan PSDM, kepada Khafidlotun Muslikhah, Erryan Sazany, Rachmad Akbar, Fauria Bisara, Dyah Inastra D., Rama Dwiyana P., Aulia Vivansyah A., Riska Fadilla, Fauzan Helmi S., Ginanjar Ibnu S., Novian Habibie, Novela Spalo, Amni Alfira, Nilamsari Putri U., Kamila Dini N., Raditya Herwando, Muhammad Meisza, Ghaisani Kusumo W. dan Dimash Narendra yang senantiasa menginspirasi penulis dan memberikan semangat dalam penulisan skripsi ini.
	\item Adkesma Se-UI 2013, kepada Moh Amar K.U., Rizki Rahmawati, Ahmad Zuhdi, Ghina Sonia F., Fera Gustina P., Utri Marliana D., M. Zaky Abdullah, Afina Hadari, Jessica Ratna S., Sistia Fitriana, Emma Septiana, Algadri Muhammad, Vizzi A.F. Nasution dan Taufik Hamzah yang selalu menyemangati, menemani hari penulis serta saling memberikan candaan segar setiap saat.
	\item Keluarga Lab TA 1229, karena diksusi, debat, \f{sharing}, \f{brainstorming} dan obrolan segar sangat berarti pada saat masa-masa pengerjaan tugas akhir.
	\item Kelompok PPL C04 dan Penghuni Microsoft Innovation Center UI, kepada M. Anas Zakaria, Farhan Syakir, Roland Raymond D., Yoga Widyakrisna, Sangadji Prabowo dan Jundi Ahmad A. yang menemani malam-malam gelap penuh kesenduan dalam pengerjaan baik tugas PPL ataupun penulisan skripsi.
	\item POMDA Fasilkom UI, kepada ibu Dina Chahyati atas bimbingan dunia perkulihan dan ibu Siska Utoyo selaku ibu asuh penulis ketika awal perkuliahan. Jasa POMDA tidak akan penulis lupakan dan akan terus dilanjutkan perjugannya oleh penulis.
	\item Grup KOMPAS teman-teman SMA yang selalu asyik untuk berdiskusi, bercanda dan bermain bersama.
	\item PT. MUCglobal, PT. Integral Data Prima dan mas Arie Widodo yang rela mengorbankan waktunya untuk membantu penulis dalam proses penelitian skripsi.
	\item Dan tak lupa untuk teman satu angkatan penulis KAWUNG Fasilkom UI 2011 yang banyak berkontribusi melahirkan pemikiran dan dinamika dalam hidup penulis.
\end{enumerate}
Penulis menyadari bahwa dalam penulisan skripsi ini masih terdapat banyak kekurangan. Oleh karena itu, saran dan kritik yang membangun dibutuhkan agar penelitian yang dilaksanakan dapat lebih baik lagi. Dengan selesainya skripsi ini, diharap dapat memberikan kontribusi yang bermanfaat untuk perkembangan ilmu pengetahuan dan teknologi khususnya dunia \f{e-government} di Indonesia.
\vspace*{0.1cm}
\begin{flushright}
Depok, 15 Juni 2015\\[0.1cm]
\vspace*{1cm}
\penulis

\end{flushright}