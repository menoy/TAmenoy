%
% Halaman Abstract
%
% @author  Andreas Febrian
% @version 1.00
%

	\chapter*{ABSTRACT}

\vspace*{0.2cm}

\noindent \begin{tabular}{l l p{11.0cm}}
	Name&: & \penulis \\
	Program&: & \programEng \\
	Title&: & \judulInggris \\
\end{tabular} \\ 

\vspace*{0.5cm}

\noindent 
Internet's technology development in the world has support the birth of e-government service for public. More than a decade Indonesia's e-government had been established. However, until now this e-government usability and user experiences aspect quality not yet known. Then we evaluated e-government website usability aspect to Indonesia's Tax Department as one of the e-government provider. The evaluation conducted by comparing usability heuristic, user satisfaction and g-quality aspect which is calculated with efficiency and effectiveness to India's tax department website. After the evaluation, it is known that Indonesia's Tax Department did not satisfy a good e-government website. From these evaluations, the writer then develop a prototype as a website improvement recommendation to Indonesia's tax department.

\vspace*{0.2cm}

\noindent Keywords: \\ 
\noindent E-government, g-Quality, User Satisfaction, Usability, User Experience, Usability Testing, Tax Department, Web-based\\

\newpage