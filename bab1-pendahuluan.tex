%-----------------------------------------------------------------------------%
\chapter{\babSatu}
%-----------------------------------------------------------------------------%

Pada bagian ini akan dijabarkan penjelasan umum mengenai penelitian yang dilakukan. Penjelasan ini meliputi latar belakang penelitian, permasalahan, tujuan penelitian, batasan penelitian, metode penelitian dan sistematika penulisan.

%-----------------------------------------------------------------------------%
\section{Latar Belakang}
%-----------------------------------------------------------------------------%
Perkembangan zaman yang semakin pesat mendorong manusia untuk lebih mandiri dan memiliki wawasan yang lebih luas. Seiring dengan kemajuan teknologi yang pesat, kebutuhan akan informasi oleh masyarakat semakin meningkat. Banyak sumber pengetahuan dan informasi yang muncul serta dapat diakses secara bebas oleh masyarakat umum.
\newline\\
Perkembangan teknologi informasi dan komunikasi akan membuka peluang dan tantangan untuk menciptakan, mengakses, mengolah dan memanfaatkan informasi secara tepat dan akurat (Hasibua, Z., 2007).  
\newline\\
Penyediaan informasi tentunya tidak hanya dilakukan oleh orang tertentu saja. Dalam akses penyebaran informasi pemerintah juga turut andil dalam masalah ketersediaan informasi. Dari dasar teknologi informasi yang berkembang dan kebutuhannya dirasakan secara langsung oleh masyarakat munculah perkembangan teknologi E-Government di lingkungan pemerintah Republik Indonesia. 
\newline\\
Bersama dengan perkembangan informasi, masyarakat menuntut pemerintah agar melaksanakan keterbukaan informasi secara struktural dan cepat. Direktorat yang ada pada Kementrian Republik Indonesia mulai berlomba dalam membuat website E-Government sesuai dengan Instruksi Presiden RI No. 3 Tahun 2003 tentang kebijakan dan strategi nasional pengembangan E-Government. Tak luput dari perhatian Direktorat Jenderal Pajak selaku direktorat dibawah Kementrian Keuangan Republik Indonesia juga menerapkan E-Government dengan membuat website yang beralamat di www.pajak.go.id . 
\newline\\
Penggunaan website dari direktorat jenderal pajak ini sangat penting dalam memenuhi kebutuhan informasi terkait pajak yang ada di Republik Indonesia. Dengan dasar tersebut Penyusun ingin melakukan penelitian mengenai \f{usability} dan fungsionalitas dari website yang dimiliki direktorat jenderal pajak ini.
\newline\\
Menurut \citeauthor{book.buyya} terdapat 3 buah contoh untuk membuat enumerate pada latex \citep{book.buyya}: 
\begin{enumerate}
\item Makan
\item Minum
\end{enumerate}\paragraph{}

Menurut \cite{ppt.ecmwf}, pemodelan yang sama apabila dijalankan dengan komputer \f{Dual Core} maka akan membutuhkan waktu 1 tahun dengan asumsi memori yang dibutuhkan cukup \citep{ppt.ecmwf}.

%-----------------------------------------------------------------------------%
\section{Perumusan Masalah}
%-----------------------------------------------------------------------------%
Pada penelitian kali ini diangkat permasalahan mengenai kemudahan dan kenyamanan penggunaan sistem website melalui tampilan antarmuka serta \f{user experience}. Kemudahan dan kenyamanan tersebut ditujukan untuk penggunaan suatu sistem website informasi pemerintah yang dapat memberikan akses yang mudah dan memperlancar proses tukar informasi antara pemerintah dengan masyarakat.

%-----------------------------------------------------------------------------%
\section{Tujuan Penelitian}
%-----------------------------------------------------------------------------%
Pengguna sistem website pemerintah khususnya website Direktorat Jenderal Pajak diharapkan merasa nyaman dan mudah dalam penggunaan website tersebut. Faktor yang mempengaruhi kemudahan serta kenyamanan pengguna dalam menggunakan sistem website adalah tampilan antarmuka serta \f{user experience}. Untuk itu, diperlukan suatu panduan dalam mendesain sistem website pemerintahan.
\newline\\
Tujuan yang ingin dicapai dari penelitian ini adalah:
\begin{enumerate}
\item Merancang instrumen perbandingan untuk penilaian aspek \f{usability} dari website \f{E-Government}.
\item Mengembangkan prototype website \f{E-Government} dari hasil penerapan instrumen yang telah dirancang.
\item Merancang instrumen dari pengembangan \f{prototype} website \f{E-Government} untuk membandingkan aspek \f{usability}.
\item Menerapkan instrumen yang telah dirancang untuk menilai aspek usability dari website direktorat jenderal pajak sebagai salah satu website \f{E-Government}.
\end{enumerate}
%-----------------------------------------------------------------------------%
\section{Batasan Penelitian}
%-----------------------------------------------------------------------------%
Dalam penelitian yang dilakukan, digunakan salah satu website \f{E-Government} milik pemerintah Indonesia yang dibandingkan dengan website \f{E-Government} milik pemerintah India sebagai alat uji. Penelitian sistem interaksi serta \f{user experience} terhadap website ini dilakukan dengan cara melakukan survei dan mengembangkan \f{prototype} website dari hasil survei. Target responden dari survei penelitian adalah konsultan pajak dan masyarakat terwajib pajak di daerah Jabodetabek.
%-----------------------------------------------------------------------------%
\section{Metode Penelitian}
%-----------------------------------------------------------------------------%
Metodologi Penelitian yang dilakukan adalah sebagai berikut:
\begin{enumerate}
	\item Studi Literatur \\
	Pada tahap ini dilakukan studi literatur terhadap materi-materi terkait dengan website E-Government seperti E-Government, sistem interaksi web, pengembangan web, dan user experience.
	\item Pembuatan Hipotesis Awal \\
	Tahap ini meliputi pembuatan hipotesis awal mengenai penilaian terhadap alat yang digunakan dalam melakukan usability testing berdasarkan studi literatur yang telah dilakukan sebelumnya.
	\item Perancangan Instrumen \\
	Tahap ini meliputi perancangan instrumen yang akan digunakan dalam menilai website E-Government. Instrumen berupa kuesioner berisi pertanyaan serta skenario penilaian.
	\item Pengumpulan Data \\
	Pada tahap ini dilakukan usability testing kepada beberapa responden sebagai pengguna dari website E-Government yang diuji. Pengujian dilakukan secara dua tahap dengan memberikan pertanyaan-pertanyaan yang telah disusun untuk mendapatkan penilaian dari pengguna umum.
	\item Pengembangan \f{Prototype} Website \\
	Pada tahap ini akan dilakukan pengembangan sebuah prototype website E-Government yang dikembangkan berdasarkan analisis data dari pengumpulan data tahap pertama terhadap pengguna umum website E-Govenrment.
	\item Analisis Data \\
	Pada tahap ini ditarik kesimpulan berdasarkan data yang telah diperoleh dari tahap pengumpulan data. Kesimpulan yang dihasilkan diperlukan untuk membuktikan hipotesis awal serta mencantumkan masukan dari responden.
\end{enumerate}
%-----------------------------------------------------------------------------%
\section{Sistematika Penulisan}
%-----------------------------------------------------------------------------%
Sistematika penulisan laporan adalah sebagai berikut:
\begin{enumerate}
	\item BAB 1 \babSatu \\
	Bab ini menjelaskan latar belakang penelitian, permasalahan, tujuan penelitian, batasan penelitian, metode penelitian, serta sistematika penulisan. 
	\item Bab 2 \babDua \\
	Bab ini menjelaskan hasil studi literatur mengenai landasan teori yang digunakan dalam melakukan penelitian.
	\item Bab 3 \babTiga \\
	Bab ini menjelaskan mengenai metodologi penelitian yang dilakukan, meliputi tahapan penelitian, model penelitian, isntrumen penelitian, hipotesis awal serta teknik pengolahan data. Hipotesis pada bab ini disusun berdasarkan buku \f{E-Government for Good Governance in Developing Country} (Kettani, Driss dan Bernard Moulin, 2014). Bab ini juga menjelaskan metode pengumpulan data yang dilakukan.
	\item Bab 4 \babEmpat \\
	Pada bab ini dijelaskan mengenai data yang telah diperoleh dan metode penarikan kesimpulan dari data responden yang telah dikumpulkan. Kesimpulan yang dihasilkan akan menentukan penilaian terhadap hipotesis awal yang telah dibuat sebelumnya.
	\item Bab 5 \babLima \\
	\item Bab 6 \babEnam \\
	Bab ini berisi mengenai kesimpulan akhir dari penelitian yang dilakukan serta saran untuk penelitian selanjutnya.
\end{enumerate}
