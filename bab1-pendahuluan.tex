%-----------------------------------------------------------------------------%
\chapter{\babSatu}
%-----------------------------------------------------------------------------%

Pada bagian ini akan dijabarkan penjelasan umum mengenai penelitian yang dilakukan. Penjelasan ini meliputi latar belakang penelitian, permasalahan, tujuan penelitian, batasan penelitian dan sistematika penulisan.

%-----------------------------------------------------------------------------%
\section{Latar Belakang}
%-----------------------------------------------------------------------------%
Perkembangan zaman yang semakin pesat mendorong manusia untuk lebih mandiri dan memiliki wawasan yang lebih luas. Seiring dengan kemajuan teknologi, kebutuhan akan informasi oleh masyarakat semakin meningkat. Banyak sumber pengetahuan dan informasi yang muncul serta dapat diakses secara bebas oleh masyarakat umum.
\newline\\
Perkembangan teknologi informasi dan komunikasi akan membuka peluang dan tantangan untuk menciptakan, mengakses, mengolah dan memanfaatkan informasi secara tepat dan akurat \citep{buku.hasibua}. Penyediaan informasi tentunya tidak hanya dilakukan oleh orang tertentu saja. Dalam akses penyebaran informasi pemerintah juga turut andil dalam masalah ketersediaan informasi. Dari dasar teknologi informasi yang berkembang dan kebutuhannya dirasakan secara langsung oleh masyarakat, muncullah perkembangan teknologi \f{e-government} di lingkungan pemerintah Republik Indonesia. 
\newline\\
Bersama dengan perkembangan informasi, masyarakat menuntut pemerintah agar melaksanakan keterbukaan informasi secara struktural dan cepat. Direktorat yang ada pada Kementrian Republik Indonesia mulai berlomba dalam membuat \f{website e-government} sesuai dengan Instruksi Presiden RI No. 6 Tahun 2001 tentang kebijakan dan strategi nasional pengembangan \f{e-government}. Tak luput dari perhatian Direktorat Jenderal Pajak selaku direktorat dibawah Kementrian Keuangan Republik Indonesia juga menerapkan \f{e-government} dengan membuat \f{website} yang beralamat di www.pajak.go.id . 
\newline\\
Penggunaan \f{website} dari Direktorat Denderal Pajak (DJP) ini sangat penting dalam memenuhi kebutuhan informasi terkait pajak yang ada di Republik Indonesia. Dengan berbagai macam informasi yang dikeluarkan oleh pihak DJP persebaran arus informasi perpajakan semakin bervariatif dan pelayanan terkait perpajakan terbantu. Berkat pencanangan \f{e-government} Indonesia dari tahun 2001 tersebut, Indonesia menduduki posisi 106 dunia dalam peringkat \f{e-government} tahun 2014\footnote{\url{http://unpan3.un.org/egovkb/en-us/Data-Center}}. Selain pemeringkatan \f{e-government} dari \f{United Nations Public Administration Network} terdapat juga pemeringkatan khusus dari \f{alexa.com} yang menempatkan \f{website} direktorat jenderal pajak Indonesia pada posisi 311 dari seluruh \f{website} umum yang ada di Indonesia\footnote{\url{http://www.alexa.com/siteinfo/pajak.go.id}}. Pemeringkatan tersebut menunjukkan bahwa \f{website} Direktorat Jenderal Pajak Indonesia memiliki tingkat akses yang cukup tinggi dari masyarakat. 
\newline\\
Berbeda halnya dengan negara India yang sama-sama tergolong dalam negara dengan populasi terbanyak di dunia \footnote{\url{http://statistik.ptkpt.net/_a.php?_a=area&info1=6}}, India baru mulai memperkenalkan sistem \f{e-government} secara formal dan efektif pada tahun 2006\footnote{\url{http://arc.gov.in/11threp/arc_11threport_ch4.pdf}}. Ada hal yang menjadi perhatian, India telah mengembangkan sistem \f{e-government} khususnya pada bidang perpajakan secara internal dimulai dari tahun 2001. Walau demikian India menempati peringkat \f{e-government} 118 dunia pada tahun 2014. \f{Website} DJP India juga menduduki peringkat 998 pada pemeringkatan alexa.com pada lingkup website negara India sendiri\footnote{\url{http://www.alexa.com/siteinfo/incometaxindia.gov.in}}.
\newline\\
Dari keterangan sebelumnya diketahui bahwa Indonesia dan India termasuk menerapkan kebijakkan \f{e-government} di tahun yang sama yaitu 2001 serta tergolong kedalam populasi penduduk yang banyak di dunia. Selain itu, \f{United Nations Public Administration Network} menggolongkan Indonesia dan India pada negara berkembang dalam pemeringkatan \f{website e-government} yang terlingkup dalam kawasan benua Asia. Dari hasil pemeringkatan yang diterangkan telah diketahui bahwa \f{website e-government} Indonesia memiliki peringkat yang lebih tinggi dibanding dengan India. Namun, jika dilihat dari pemeringkatan secara global untuk website DJP oleh \f{alexa.com, website} DJP Indonesia mendapat posisi 25142 peringkat yang lebih rendah dari DJP India yang berada pada posisi 14612. Kemudian muncul \f{website} di Indonesia yang memiliki fungsi tak jauh berbeda dengan \f{website} DJP Indonesia, walau hanya sebatas informasi bukan terkait layanan perpajakan pribadi seperti \f{website ortax.org}.
\newline\\
Dari pemeringkatan perkembangan \f{e-government} yang tergolong lebih unggul dari India, Indonesia seakan melupakan salah satu aspek dalam pengembangan \f{e-government} seperti penelitian terkait aspek \f{usability} yang dilakukan oleh \citet{paper.dahlan}. \citeauthor{paper.dahlan} mengemukakan bahwa \f{website e-government} di Indonesia memiliki tingkat \f{usability} dan kepuasan pengguna yang kurang memuaskan. Penelitian dari \citeauthor{paper.dahlan} dilakukan dengan menguji beberapa \f{website e-government} di daerah. Sementara itu penelitian pada \f{website e-government} yang sering diakses oleh pengguna seperti \f{website} DJP Indonesia masih belum banyak dilakukan. 
\newline\\
Dengan landasan kemiripan penerapan e-government, golongan negara berkembang serta negara dengan populasi terbanyak. Penulis ingin melakukan penelitian mengenai \f{usability website e-government}  dari \f{website} yang dimiliki direktorat jenderal pajak Indonesia dan membandingkannya dengan {website} milik direktorat jenderal pajak India.
%-----------------------------------------------------------------------------%
\section{Perumusan Masalah}
%-----------------------------------------------------------------------------%
Sesuai dengan latar belakang yang sudah disampaikan,  penelitian ini mengangkat permasalahan mengenai \f{usability} dari \f{website e-government} melalui tampilan antarmuka dan \f{user experience} yang belum memenuhi harapan. Berikut merupakan rumusan masalah pada penelitian ini.
\begin{enumerate}
	\item Apakah \f{website} \f{e-government} khususnya Direktorat Jenderal Pajak Indonesia sudah memenuhi standar \f{usability e-government} yang baik?
	\item Apakah hasil evaluasi \f{usability e-government} \f{website} Direktorat Jenderal Pajak Indonesia lebih baik dari \f{website} Direktorat Jenderal Pajak India?
\end{enumerate}

%-----------------------------------------------------------------------------%
\section{Tujuan Penelitian}
%-----------------------------------------------------------------------------%
Sesuai dengan permasalahan yang telah dijelaskan sebelumnya. Penulis menjabarkan tujuan yang ingin dicapai dari penelitian ini sebagai berikut:
\begin{enumerate}
\item Melaksanakan \f{usability testing} untuk menilai aspek \f{usability} dan \f{g-quality} dari \f{website} direktorat jenderal pajak sebagai salah satu \f{website} \f{e-government}.
\item Menganalisis data untuk mendapatkan informasi perbandingan mengenai aspek 
\f{usability} dan rekomendasi pengembangan \f{user interface} dari responden.
\item Mengembangkan \f{high-fidelity prototype} dari hasil penerapan instrumen yang telah dirancang sebagai bentuk masukan untuk \f{website} \f{e-government} terkait.
\end{enumerate}
%-----------------------------------------------------------------------------%
\section{Batasan Penelitian}
%-----------------------------------------------------------------------------%
Dalam penelitian ini, terdapat beberapa batasan yang diterapkan. Berikut merupakan batasan dalam penelitian ini.
\begin{enumerate}
	\item Pengujian pada \f{website e-government} dilakukan menggunakan \f{web browser} pada PC.
	\item Penelitian dilakukan dengan evaluasi \f{usability e-government} terhadap \f{website e-government} dengan cara melakukan \ust, analisa data dan mengembangkan \f{prototype} dari hasil \ust.
	\item Target responden dari penelitian ini dikhususkan pada konsultan pajak, pekerja di bidang pajak dan mahasiswa administrasi perpajakan di daerah Jabodetabek.
	\item Sistem yang dievaluasi hanya bentuk interface dari \f{website} portal dan aplikasi \f{e-filling}.
\end{enumerate}
%-----------------------------------------------------------------------------%
\section{Sistematika Penulisan}
%-----------------------------------------------------------------------------%
Sistematika penulisan laporan adalah sebagai berikut:
\begin{enumerate}
	\item BAB 1 \babSatu \\
	Bab ini menjelaskan latar belakang penelitian, permasalahan, tujuan penelitian, batasan penelitian serta sistematika penulisan. 
	\item Bab 2 \babDua \\
	Bab ini menjelaskan hasil studi literatur mengenai landasan teori yang digunakan dalam melakukan penelitian.
	\item Bab 3 \babTiga \\
	Bab ini menjelaskan mengenai metodologi penelitian yang dilakukan, meliputi tahapan penelitian, model penelitian, isntrumen penelitian, pelaksanaan penelitian serta teknik pengolahan data. Bab ini juga menjelaskan metode pengumpulan data yang dilakukan.
	\item Bab 4 \babEmpat \\
	Pada bab ini dijelaskan mengenai data yang telah diperoleh dari penelitian. Kemudian data dianalisa dan dibahas keterkaitannya dengan teori sehingga memunculkan suatu rumusan.
	\item Bab 5 \babLima \\
	Bab ini berisi mengenai pengembangan \f{high-fidelity prototype} dari hasil analisis \f{usability testing} dan rekomendasi pengguna. 
	\item Bab 6 \babEnam \\
	Bab ini berisi mengenai kesimpulan akhir dari penelitian yang dilakukan serta saran untuk penelitian selanjutnya.
\end{enumerate}
