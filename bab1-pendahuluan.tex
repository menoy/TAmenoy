%-----------------------------------------------------------------------------%
\chapter{\babSatu}
%-----------------------------------------------------------------------------%

Pada bagian ini akan dijabarkan penjelasan umum mengenai penelitian yang dilakukan. Penjelasan ini meliputi latar belakang penelitian, permasalahan, tujuan penelitian, batasan penelitian, metode penelitian dan sistematika penulisan.

%-----------------------------------------------------------------------------%
\section{Latar Belakang}
%-----------------------------------------------------------------------------%
Perkembangan zaman yang semakin pesat mendorong manusia untuk lebih mandiri dan memiliki wawasan yang lebih luas. Seiring dengan kemajuan teknologi yang pesat, kebutuhan akan informasi oleh masyarakat semakin meningkat. Banyak sumber pengetahuan dan informasi yang muncul serta dapat diakses secara bebas oleh masyarakat umum.
\newline\\
Perkembangan teknologi informasi dan komunikasi akan membuka peluang dan tantangan untuk menciptakan, mengakses, mengolah dan memanfaatkan informasi secara tepat dan akurat (Hasibua, Z., 2007).  
\newline\\
Penyediaan informasi tentunya tidak hanya dilakukan oleh orang tertentu saja. Dalam akses penyebaran informasi pemerintah juga turut andil dalam masalah ketersediaan informasi. Dari dasar teknologi informasi yang berkembang dan kebutuhannya dirasakan secara langsung oleh masyarakat munculah perkembangan teknologi E-Government di lingkungan pemerintah Republik Indonesia. 
\newline\\
Bersama dengan perkembangan informasi, masyarakat menuntut pemerintah agar melaksanakan keterbukaan informasi secara struktural dan cepat. Direktorat yang ada pada Kementrian Republik Indonesia mulai berlomba dalam membuat website E-Government sesuai dengan Instruksi Presiden RI No. 3 Tahun 2003 tentang kebijakan dan strategi nasional pengembangan E-Government. Tak luput dari perhatian Direktorat Jenderal Pajak selaku direktorat dibawah Kementrian Keuangan Republik Indonesia juga menerapkan E-Government dengan membuat website yang beralamat di www.pajak.go.id . 
\newline\\
Penggunaan website dari direktorat jenderal pajak ini sangat penting dalam memenuhi kebutuhan informasi terkait pajak yang ada di Republik Indonesia. Dengan dasar tersebut Penyusun ingin melakukan penelitian mengenai \f{usability} dan fungsionalitas dari website yang dimiliki direktorat jenderal pajak ini.
\newline\\
Menurut \citeauthor{book.buyya} terdapat 3 buah contoh untuk membuat enumerate pada latex \citep{book.buyya}: 
\begin{enumerate}
\item Makan
\item Minum
\end{enumerate}\paragraph{}

Menurut \cite{ppt.ecmwf}, pemodelan yang sama apabila dijalankan dengan komputer \f{Dual Core} maka akan membutuhkan waktu 1 tahun dengan asumsi memori yang dibutuhkan cukup \citep{ppt.ecmwf}.

%-----------------------------------------------------------------------------%
\section{Perumusan Masalah}
%-----------------------------------------------------------------------------%
Pada bagian ini akan dijelaskan mengenai definisi permasalahan yang dihadapi dan ingin diselesaikan serta asumsi dan batasan yang digunakan dalam menyelesaikannya.

%-----------------------------------------------------------------------------%
\section{Tujuan dan Manfaat Penelitian}
%-----------------------------------------------------------------------------%
Dibawah ini adalah contoh itemize : 
\begin{itemize}
\item Terimplementasinya .
\item Menyelesaikan masalah .
\end{itemize}
\paragraph{}

%-----------------------------------------------------------------------------%
\section{Tahapan Penelitian}
%-----------------------------------------------------------------------------%
\todo{Tuliskan tujuan penelitian.}
%-----------------------------------------------------------------------------%
\section{Ruang Lingkup Penelitian}
%-----------------------------------------------------------------------------%

%-----------------------------------------------------------------------------%
\section{Sistematika Penulisan}
%-----------------------------------------------------------------------------%
Sistematika penulisan laporan adalah sebagai berikut:
\begin{itemize}
	\item Bab 1 \babSatu \\
	\item Bab 2 \babDua \\
	\item Bab 3 \babTiga \\
	\item Bab 4 \babEmpat \\
	\item Bab 5 \babLima \\
	\item Bab 6 \babEnam \\
	\item Bab 7 \babTujuh \\
\end{itemize}

\todo{Tambahkan penjelasan singkat mengenai isi masing-masing bab.}