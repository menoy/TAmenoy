%-----------------------------------------------------------------------------%
\chapter{\babSatu}
%-----------------------------------------------------------------------------%

Pada bagian ini akan dijabarkan penjelasan umum mengenai penelitian yang dilakukan. Penjelasan ini meliputi latar belakang penelitian, permasalahan, tujuan penelitian, batasan penelitian dan sistematika penulisan.

%-----------------------------------------------------------------------------%
\section{Latar Belakang}
%-----------------------------------------------------------------------------%
Perkembangan zaman yang semakin pesat mendorong manusia untuk lebih mandiri dan memiliki wawasan yang lebih luas. Seiring dengan kemajuan teknologi yang pesat, kebutuhan akan informasi oleh masyarakat semakin meningkat. Banyak sumber pengetahuan dan informasi yang muncul serta dapat diakses secara bebas oleh masyarakat umum.
\newline\\
Perkembangan teknologi informasi dan komunikasi akan membuka peluang dan tantangan untuk menciptakan, mengakses, mengolah dan memanfaatkan informasi secara tepat dan akurat (Hasibua, Z., 2007).  
\newline\\
Penyediaan informasi tentunya tidak hanya dilakukan oleh orang tertentu saja. Dalam akses penyebaran informasi pemerintah juga turut andil dalam masalah ketersediaan informasi. Dari dasar teknologi informasi yang berkembang dan kebutuhannya dirasakan secara langsung oleh masyarakat munculah perkembangan teknologi E-Government di lingkungan pemerintah Republik Indonesia. 
\newline\\
Bersama dengan perkembangan informasi, masyarakat menuntut pemerintah agar melaksanakan keterbukaan informasi secara struktural dan cepat. Direktorat yang ada pada Kementrian Republik Indonesia mulai berlomba dalam membuat \f{website} E-Government sesuai dengan Instruksi Presiden RI No. 3 Tahun 2003 tentang kebijakan dan strategi nasional pengembangan E-Government. Tak luput dari perhatian Direktorat Jenderal Pajak selaku direktorat dibawah Kementrian Keuangan Republik Indonesia juga menerapkan E-Government dengan membuat \f{website} yang beralamat di www.pajak.go.id . 
\newline\\
Penggunaan \f{website} dari direktorat jenderal pajak ini sangat penting dalam memenuhi kebutuhan informasi terkait pajak yang ada di Republik Indonesia. Dengan dasar tersebut Penyusun ingin melakukan penelitian mengenai \f{usability} dan fungsionalitas dari \f{website} yang dimiliki direktorat jenderal pajak ini.
%-----------------------------------------------------------------------------%
\section{Perumusan Masalah}
%-----------------------------------------------------------------------------%
Pada penelitian kali ini diangkat permasalahan mengenai kemudahan dan kenyamanan penggunaan sistem \f{website} melalui tampilan antarmuka serta \f{user experience} yang belum memenuhi harapan. Kemudahan dan kenyamanan tersebut ditujukan untuk penggunaan suatu sistem \f{website} informasi pemerintah yang dapat memberikan akses yang mudah dan memperlancar proses tukar informasi antara pemerintah dengan masyarakat.

%-----------------------------------------------------------------------------%
\section{Tujuan Penelitian}
%-----------------------------------------------------------------------------%
Pengguna sistem \f{website} pemerintah khususnya \f{website} Direktorat Jenderal Pajak diharapkan merasa nyaman dan mudah dalam penggunaan \f{website} tersebut. Faktor yang mempengaruhi kemudahan serta kenyamanan pengguna dalam menggunakan sistem \f{website} adalah tampilan antarmuka serta \f{user experience}. Untuk itu, diperlukan suatu panduan dalam mendesain sistem \f{website} pemerintahan.
\newline\\
Tujuan yang ingin dicapai dari penelitian ini adalah:
\begin{enumerate}
\item Merancang instrumen perbandingan untuk penilaian aspek \f{usability} dari \f{website} \f{E-Government}.
\item Mengembangkan prototype \f{website} \f{E-Government} dari hasil penerapan instrumen yang telah dirancang.
\item Merancang instrumen dari pengembangan \f{prototype} \f{website} \f{E-Government} untuk membandingkan aspek \f{usability}.
\item Menggunakan instrumen yang telah dirancang untuk menilai aspek usability dari \f{website} direktorat jenderal pajak sebagai salah satu \f{website} \f{E-Government}.
\item Mengembangkan \f{prototype} sebagai bentuk masukkan untuk \f{website} \f{e-government} terkait.
\end{enumerate}
%-----------------------------------------------------------------------------%
\section{Batasan Penelitian}
%-----------------------------------------------------------------------------%
Dalam penelitian yang dilakukan, digunakan salah satu \f{website} \f{E-Government} milik pemerintah Indonesia yang dibandingkan dengan \f{website} \f{E-Government} milik pemerintah India sebagai alat uji. Penelitian sistem interaksi serta \f{user experience} terhadap \f{website} ini dilakukan dengan cara melakukan survei dan mengembangkan \f{prototype} \f{website} dari hasil survei. Target responden dari survei penelitian adalah konsultan pajak dan mahasiswa administrasi perpajakan di daerah Jabodetabek. Sistem yang dievaluasi hanya bentuk interface dari \f{website} portal dan aplikasi \f{e-filling}.
%-----------------------------------------------------------------------------%
\section{Sistematika Penulisan}
%-----------------------------------------------------------------------------%
Sistematika penulisan laporan adalah sebagai berikut:
\begin{enumerate}
	\item BAB 1 \babSatu \\
	Bab ini menjelaskan latar belakang penelitian, permasalahan, tujuan penelitian, batasan penelitian, metode penelitian, serta sistematika penulisan. 
	\item Bab 2 \babDua \\
	Bab ini menjelaskan hasil studi literatur mengenai landasan teori yang digunakan dalam melakukan penelitian.
	\item Bab 3 \babTiga \\
	Bab ini menjelaskan mengenai metodologi penelitian yang dilakukan, meliputi tahapan penelitian, model penelitian, isntrumen penelitian, hipotesis awal serta teknik pengolahan data. Hipotesis pada bab ini disusun berdasarkan buku \f{E-Government for Good Governance in Developing Country} (Kettani, Driss dan Bernard Moulin, 2014). Bab ini juga menjelaskan metode pengumpulan data yang dilakukan.
	\item Bab 4 \babEmpat \\
	Pada bab ini dijelaskan mengenai data yang telah diperoleh dan metode penarikan kesimpulan dari data responden yang telah dikumpulkan. Kesimpulan yang dihasilkan akan menentukan penilaian terhadap hipotesis awal yang telah dibuat sebelumnya.
	\item Bab 5 \babLima \\
	Bab ini berisi mengenai pengembangan \f{prototype user interface} dari hasil analisis \f{usability testing} dan rekomendasi pengguna. Setelah dikembangkan, \f{prototype} dievaluasi oleh pengguna.
	\item Bab 6 \babEnam \\
	Bab ini berisi mengenai kesimpulan akhir dari penelitian yang dilakukan serta saran untuk penelitian selanjutnya.
\end{enumerate}
