%-----------------------------------------------------------------------------%
\chapter{\babSatu}
%-----------------------------------------------------------------------------%

Pada bagian ini akan dijabarkan penjelasan umum mengenai penelitian yang dilakukan. Penjelasan ini meliputi latar belakang penelitian, permasalahan, tujuan penelitian, batasan penelitian dan sistematika penulisan.

%-----------------------------------------------------------------------------%
\section{Latar Belakang}
%-----------------------------------------------------------------------------%
Perkembangan zaman yang semakin pesat mendorong manusia untuk lebih mandiri dan memiliki wawasan yang lebih luas. Seiring dengan kemajuan teknologi, kebutuhan akan informasi oleh masyarakat semakin meningkat. Banyak sumber pengetahuan dan informasi yang muncul serta dapat diakses secara bebas oleh masyarakat umum.
\newline\\
Perkembangan teknologi informasi dan komunikasi akan membuka peluang dan tantangan untuk menciptakan, mengakses, mengolah dan memanfaatkan informasi secara tepat dan akurat \citep{buku.hasibua}. Penyediaan informasi tentunya tidak hanya dilakukan oleh orang tertentu saja. Dalam akses penyebaran informasi pemerintah juga turut andil dalam masalah ketersediaan informasi. Dari dasar teknologi informasi yang berkembang dan kebutuhannya dirasakan secara langsung oleh masyarakat, munculah perkembangan teknologi \f{e-government} di lingkungan pemerintah Republik Indonesia. 
\newline\\
Bersama dengan perkembangan informasi, masyarakat menuntut pemerintah agar melaksanakan keterbukaan informasi secara struktural dan cepat. Direktorat yang ada pada Kementrian Republik Indonesia mulai berlomba dalam membuat \f{website} \f{e-government} sesuai dengan Instruksi Presiden RI No. 6 Tahun 2001 tentang kebijakan dan strategi nasional pengembangan \f{e-government}. Tak luput dari perhatian Direktorat Jenderal Pajak selaku direktorat dibawah Kementrian Keuangan Republik Indonesia juga menerapkan \f{e-government} dengan membuat \f{website} yang beralamat di www.pajak.go.id . 
\newline\\
Penggunaan \f{website} dari Direktorat Denderal Pajak (DJP) ini sangat penting dalam memenuhi kebutuhan informasi terkait pajak yang ada di Republik Indonesia. Dengan berbagai macam informasi yang dikeluarkan oleh pihak DJP persebaran arus informasi perpajakan semakin bervariatif dan pelayanan terkait perpajakan terbantu. Berkat pencanangan \f{e-government} Indonesia dari tahun 2001 tersebut, Indonesia menduduki posisi 97 dunia dalam peringkat \f{e-government} tahun 2012\footnote{\url{http://unpan3.un.org/egovkb/Portals/egovkb/Documents/un/2012-Survey/Chapter-1-World-e-government-rankings.pdf}}. 
\newline\\
Berbeda halnya dengan negara India yang sama-sama memiliki jumlah populasi terbanyak didunia, India baru mulai memperkenalkan sistem \f{e-government} secara formal efektif pada tahun 2006\footnote{\url{http://arc.gov.in/11threp/arc_11threport_ch4.pdf}}. Ada hal yang menjadi perhatian, India telah mengembangkan sistem \f{e-government} perpajakan secara internal dimulai dari tahun 2001. Walau demikian India menempati peringkat \f{e-government} 125 dunia pada tahun 2012.
\newline\\
Uniknya, walau Indonesia menempati peringkat lebih tinggi dari India, tetapi banyak muncul \f{website} di Indonesia yang memiliki fungsi tak jauh berbeda dengan \f{website} DJP Indonesia, walau hanya sebatas informasi bukan terkait layanan perpajakan pribadi. Salah satu \f{website} yang menginformasikan terkait hal tersebut adalah www.ortax.org. Namun, sesuai dengan penelitian yang dilakukan oleh \citet{paper.dahlan}, \f{website e-government} Indonesia memiliki tingkat kepuasan yang kurang memuaskan.
\newline\\ 
Dengan dasar tersebut Penyusun ingin melakukan penelitian mengenai \f{usability website e-government} dari \f{website} yang dimiliki direktorat jenderal pajak ini dengan membandingkannya dengan {website} milik direktorat jenderal pajak India.
%-----------------------------------------------------------------------------%
\section{Perumusan Masalah}
%-----------------------------------------------------------------------------%
Pada penelitian kali ini diangkat permasalahan mengenai kemudahan dan kenyamanan penggunaan \f{website e-government} melalui tampilan antarmuka serta \f{user experience} yang belum memenuhi harapan. Berikut merupakan rumusan masalah pada penelitian ini.
\begin{enumerate}
	\item Apakah \f{website} \f{e-government} khususnya Direktorat Jenderal Pajak Indonesia sudah memenuhi standar \f{usability e-government} yang baik?
	\item Apakah hasil evaluasi \f{usability e-government} \f{website} Direktorat Jenderal Pajak Indonesia lebih baik dari \f{website} Direktorat Jenderal Pajak India?
\end{enumerate}

%-----------------------------------------------------------------------------%
\section{Tujuan Penelitian}
%-----------------------------------------------------------------------------%
Pengguna sistem \f{website e-government} khususnya \f{website} Direktorat Jenderal Pajak diharapkan merasa nyaman dan mudah dalam penggunaan \f{website} tersebut. Faktor yang mempengaruhi kemudahan serta kenyamanan pengguna dalam menggunakan sistem \f{website} adalah tampilan antarmuka serta \f{user experience}. Untuk itu, diperlukan suatu panduan dalam mendesain sistem \f{website} pemerintahan.
\newline\\
Tujuan yang ingin dicapai dari penelitian ini adalah:
\begin{enumerate}
\item Merancang instrumen perbandingan untuk penilaian aspek \f{usability} dari \f{website} \f{e-government}.
\item Menggunakan instrumen yang telah dirancang untuk menilai aspek \f{usability g-quality} dari \f{website} direktorat jenderal pajak sebagai salah satu \f{website} \f{e-government}.
\item Menganalisis data untuk mendapatkan informasi mengenai rekomendasi pengembangan \f{user interface} yang baik.
\item Mengembangkan \f{high-fidelity prototype} dari hasil penerapan instrumen yang telah dirancang sebagai bentuk masukkan untuk \f{website} \f{e-government} terkait.
\end{enumerate}
%-----------------------------------------------------------------------------%
\section{Batasan Penelitian}
%-----------------------------------------------------------------------------%
Dalam penelitian ini, terdapat beberapa batasan yang diterapkan. Berikut merupakan batasan dalam penelitian ini.
\begin{enumerate}
	\item Penelitian inig mengggunakan salah satu \f{website} \f{e-government} milik pemerintah Indonesia (\url{http://www.pajak.go.id/}) yang dibandingkan dengan \f{website} \f{e-government} milik pemerintah India (\url{http://incometaxindia.gov.in/}) sebagai alat uji.
	\item Pengujian pada \f{website e-government} dilakukan menggunakan \f{web browser} pada PC.
	\item Penelitian dilakukan dengan evaluasi \f{usability e-government} terhadap \f{website e-government} dengan cara melakukan survei \ust, analisa data dan mengembangkan \f{prototype} dari hasil survei.
	\item Target responden dari survei penelitian adalah konsultan pajak, pekerja di bidang pajak dan mahasiswa administrasi perpajakan di daerah Jabodetabek.
	\item Sistem yang dievaluasi hanya bentuk interface dari \f{website} portal dan aplikasi \f{e-filling}.
\end{enumerate} , .  
%-----------------------------------------------------------------------------%
\section{Sistematika Penulisan}
%-----------------------------------------------------------------------------%
Sistematika penulisan laporan adalah sebagai berikut:
\begin{enumerate}
	\item BAB 1 \babSatu \\
	Bab ini menjelaskan latar belakang penelitian, permasalahan, tujuan penelitian, batasan penelitian serta sistematika penulisan. 
	\item Bab 2 \babDua \\
	Bab ini menjelaskan hasil studi literatur mengenai landasan teori yang digunakan dalam melakukan penelitian.
	\item Bab 3 \babTiga \\
	Bab ini menjelaskan mengenai metodologi penelitian yang dilakukan, meliputi tahapan penelitian, model penelitian, isntrumen penelitian, pelaksanaan penelitian serta teknik pengolahan data. Bab ini juga menjelaskan metode pengumpulan data yang dilakukan.
	\item Bab 4 \babEmpat \\
	Pada bab ini dijelaskan mengenai data yang telah diperoleh dari penelitian. Kemudian data dianalisa dan dibahas keterkaitannya dengan teori sehingga memunculkan suatu rumusan.
	\item Bab 5 \babLima \\
	Bab ini berisi mengenai pengembangan \f{high-fidelity prototype} dari hasil analisis \f{usability testing} dan rekomendasi pengguna. 
	\item Bab 6 \babEnam \\
	Bab ini berisi mengenai kesimpulan akhir dari penelitian yang dilakukan serta saran untuk penelitian selanjutnya.
\end{enumerate}
